% -*- coding: utf-8 -*-

\input macros

\beginchapter Chapter 5. 实修

\origpageno=39
\pageno=37

{
\parindent=3pc
\noindent\hang\hangafter-2
\1\smash{\lower19pt\hbox to 0pt{\hskip-\hangindent\KT{36}虽\hfill}}%
{\KT{12}然禅修的科目多种多样},但我们强烈建议你从不间断地专注于你的呼
吸以获得一定程度的基本的定\myfootnote{concentration: 专注。}开始。在
做这个时请记住,你不是在修炼深度禅定\myfootnote{absorption: 禅定。}或
单纯的专注技巧。你是在修炼正念,因此你只需要一定程度的基本的定。你想要
的是培养正念以最终产生证悟如实真理的参悟和智慧。你想要的是如实认知身心
综合体的运作机理。你想要的是消除所有心理烦恼使你的人生真正平静和快乐。

}

没有看到事物的真相内心就无法清净。“如实知见”是如此满含深意又含糊不清的
短语。许多初学者不明白我们所说的意思,因为看似任何具有清楚视力的人应该
都可以如实地看清事物。

当我们用这个短语来指代从禅修中获得的参悟时,我们并不是指通过普通的眼睛
从表面看清事物,而是指通过智慧从它们的内在如实地明白它们。用智慧知见意
指不带着由贪、瞋、痴所产生的成见和偏颇从我们身心综合体\myfootnote{
body-mind complex}的框架内看清事物。一般情况下,当我们观察我
们\1身心综合体的运作时,我们试图忽略那些对我们来说不快的事物而执著于那些
快乐的事物。这是因为通常我们的内心受到贪、瞋、痴的影响。我们的自我、自已、
或观点挡住我们并歪曲了我们的判断。

当我们具念地观察身体的感觉时,我们不应将其与心理行为相混淆,因为身体的
感觉可以完全独立于内心而出现。比如,我们正在舒服地打坐。过了一阵之后,某
个不适的感受在我们的后背或腿部出现。我们的内心立刻感受到这个不适并围绕
着这个不适形成众多的想法。在这时,不要将不适的感受与心理行为相混淆,我们
应该将感受就单独作为感受来具念地观察。受\myfootnote{feeling: 感受。}是
七遍行心所\myfootnote{seven universal metal factors: 七种普遍存在的心理
因素。}之一。其它六种分别是:触\myfootnote{contact: 意识与对象相遇。}、想
\myfootnote{perception: 感知、联想。}、作意\myfootnote{attention: 注意。}、
定\myfootnote{concentration: 专注。}、命\myfootnote{life force: 生命力。}、
思\myfootnote{volition: 造作,意志力。}。

一段时间之后,某个特定的情绪,比如憎恨、害怕或渴望,会出现。在这段时间,
我们应如实地观察这个情绪,不要将其与其它任何事物相混淆。当我们将色
\myfootnote{form: 有形的物质世界,内六根,外六境(尘)。}、受\myfootnote%
{feeling: 苦乐舍(不苦不乐)的感受。}、想\myfootnote{perceptions: 思想,
观念的形成过程。}、行\myfootnote{mental formations: 除受、想的一切造作
的心理行为,如贪瞋痴等,是使口身意作业的直接原因。}、识\myfootnote%
{consciousness: 意识,心的主体(心王),六识(眼耳鼻舌身意)。}五蕴
\myfootnote{aggregates: 聚积,五蕴。色蕴属色法(物质现象),识蕴属心法
(意识现象),受想行蕴属于心所法(由意识形成的现象),色与识相互作用(触)
产生受想行。}混为一谈并将它们统统视为某种感受,因为感受的来源变得含糊不
清我们就被搞糊涂了。如果我们只是停留于这个感受而不将其与其它心理因素相
区分,我们对真理的证悟变得非常困难。

我们想要从无常的经验中获得证悟以克服我们的不快和无知:我们对不快的深入
认知将克服导致我们不快的贪欲,而我们对无我的证悟将克服因自我的观念所导
致的无知。为了达到这些证悟,我们通过分开看待身心开始;分别理解它们的同
时,我们也应看到它们必不可少的相联性。当我们的证悟更深刻时,我们越来越
觉察到:五蕴,身心,是协同合作的,并且\1两者相互依存的。我们能够真正理解
那个关于身体健全的盲人和视力完好的残疾人的著名比喻的含义。单独的盲人或
残疾人,他们都是受限的。但当视力完好的残疾人坐到身体健全的盲人肩膀上时,
他们一起可以不受限地自由行走并可以轻松地到达目的地。身体和内心就像这样。
身体无法独自完成任何事情;它就像圆木一样自己无法移动或做其它任何事情,
除了屈从于无常、腐败和死亡。内心如果没有身体的支撑也无法做任何事情。当
我们具念地观察身体和内心两者时,我们会看到它们一起可以完成多么美妙的事
情。

通过定点的坐禅,我们可以获得一定程度的正念。闭关静修用数天或数周的时间
静观自己的感受、感知、无数的念头、以及不同的意识状态,这种方式最终可使
我们安定且平静。但通常我们没有那么多时间停留在一个地方,将全部时间投入
到禅修中。因此,我们应找到一种方式将我们的正念应用到日常生活中以使我们
能够应对日常生活中的不测事件。

我们每天面对的事物是不可预测的。事物由多样的(内在)原因和(外在)条件而
产生,因为我们生活在一个因缘而生且无常的世界里。正念是我们的应急箱,随
时准备救急。当我们面对一个让我们觉得义愤填膺的情况时,如果我们具念地探
究自己的内心,我们将发现关于自己的苦涩的真相:比如,我们是自私的;我们是
以自我为中心的;我们是执著自我的;我们是固执己见的;我们总是认为自己是对
的其它人都是错的;我们是偏颇的;我们甚至是偏见的;而藏在这些之下的真相
是:我们并不真正地爱我们自己。这个发现虽然苦涩却是非常有价值的经验。而
从长远的角度来说,这个发现将使我们从根深蒂固的心理和灵魂的苦难中得到解
脱。

\1正念实修是百分之百的忠实于自己的修行。当我们观察自己的身心时,我们注
意到某些意识到它是令人不快的事物。因为我们不喜欢它们,我们试图排斥它
们。这些我们不喜欢的事物是什么?我们不喜欢与我们所爱的分离或与我们不爱
的相处。我们不仅将人物、处所和物质事物包括在我们喜欢与不喜欢的分类中,
还将看法、观念、信仰和决定囊括其中。我们不喜欢自然地发生在我们身上的事
情。比如,我们不喜欢变老、生病、变虚弱或展示自己的年纪,因为我们有一个
强烈的欲望保持自己的外貌。我们不喜欢别人指出我们的错误,因为我们对自己
引以为豪。我们不喜欢别人比我们更智慧,因为我们自己迷惑了自己。这些只是
我们个人贪、瞋、痴经验的部分例子。

当贪、瞋、痴在我们日常生活中显露时,我们用我们的正念来追踪它们并领悟它
们的根源。这些心理状态的根源在我们自己身上。假如我们身上没有仇恨之根,
没有人能让我们生气,正是我们身上的愤怒之根对他人的动作、语言或行为做出
反应。如果我们是具念的,我们将精勤地使用自己的智慧深入观察自己的内心。
如果我们心中没有仇恨,我们就不会对他人指出我们的缺点有所介怀。相反,我
们会对让我们注意到自己过失之人心存感激。我们必须极具智慧和正念地感谢
为帮助我们踏上自我改善之路而指出我们过失的人。我们所有人都有盲点。他
人是我们的镜子,透过他们我们可以用智慧看到自己的过失。我们应该将指出我
们的缺点之人视为挖掘出我们尚未发觉的隐藏的财宝之人,因为正是通过认识到
我们自身瑕疵的存在我们才得以改善自己。改善自己是我们\1通向人生圆满之目
标的坚持不懈的途径。在我们试图超越自身的缺陷之前,我们应该知道它们是什
么。通过克服这些弱点,也只有如此,我们方能培养深藏在我们潜意识中的高贵
品质。

这样来理解它:如果我们生病了,我们必须找出生病的原因。只有这样
我们才能得到有效的治疗。如果我们假装自己没有病,即使我们非常痛苦,我们
将永远得不到治疗。相同地,如果我们认为自己没有这些过失,我们将永远无法
清除我们修行之路上的障碍。如果我们无法看到自身的缺点,我们就需要有人为
我们指出来。当他们指出我们的过失时,我们应该像舍利弗尊者一样地感激他们,
舍利弗尊者曾说:“即使七岁的小和尚指出我的错误,我也将以对他无比尊敬的
态度来接受它们。”舍利弗尊者是一位百分百具念并且没有过失的僧人。因为他
没有一丁点的傲慢,所以他能达到这个地步。虽然我们不是阿罗汉,我们也应下
定决心以他为榜样,因为我们人生的目标就是达成他已达成的。

当然,指出我们错误的人或许自己也不是完人,但是他能够看到我们的过失就像
我们能看到他的,只有我们为他指出来他才能注意到。无论是指出他人的缺点还
是回应他人对自己缺点的指出都应具念以为之。如果一个人不具念地指出缺点并
使用不友善和严厉的语言,他不仅不会利人还会对自己以及被他指出缺点的人带
来伤害。带着憎恨说话的人无法具念且清楚地表达自己。听到恶语而感到受伤的
人会失去正念且听不进他人真正要表达的。我们应该具念地说话和具念地倾听,
通过交流使自己受益。当我们具念地交流,我们的内心将摆脱贪婪、自私、仇恨
和痴妄。

\subsectcnon 目标

作为禅修者,我们必须拥有一个目标,因为如果我们没有目标,盲目地跟随他人
的禅修指令,我们将只会在黑暗中摸索。必须要有一个我们有意并且自愿为之的
目标。在别人之前开悟、比别人更有力量或者比别人获得更多的收益,这些都不
是内观禅修者的目标。禅修者不是为了正念而彼此竞争。

我们的目标是让潜藏在我们潜意识中的高尚和善巧的品质达到圆满。这个目标有
五个要素:内心清净,克服悲愁,克服痛苦,踏上达成永远平静的正道,并且通过
追随正道而获得快乐。将这五重目标牢记心中,我们就能带着希望和自信前行。

\vfill
\subsectcnon 实修

一旦你坐下,就不要再改变坐姿直到你事先决定的结束时间。试想因为不舒服而
改变你原来的坐姿,并使用另一个姿势。那么过一会儿发生的是这个新坐姿变得
不舒服。于是你想要另一个姿势,过一会儿它又变得不舒服。这样你将不停地变
换、移动,在你坐在禅修垫上的整段时间里不断改变姿势,进而你将无法获得一
个深度的和有价值的定\myfootnote{concentration: 专注。}。因此,你必须尽
一切努力不改变你原来的姿势。我们将在第10章中讨论如何应对身体的不适。

为避免改变你的坐姿,先在禅修开始前决定你将要坐禅的时长。如果你从未进行
过禅修,坐着不动的时间不要超过二十分钟。当你不断重复地修行后,你可以渐
进地延长你坐禅的时间。打坐的时长取决于你所拥有的坐禅的时间以及你能一直
打坐而不觉得极度痛苦的时间长度。

我们不应该为达到禅修的目标制定一个时间表,我们所取得的成就取决于我们修
行的进度,修行的进度基于我们对灵性能力的理解和开发。我们必须朝着目标勤
劳且具念地不断精进,而非设定任何特定的时间来达到它。当我们已准备好时,
我们就自然而然到达那里。我们所必须做的就是为了那个达成不断地完善自己做
好准备。

\endchapter

\byebye
