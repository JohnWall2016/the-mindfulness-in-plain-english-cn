% -*- coding: utf-8 -*-

\input macros

\beginchapter Chapter 5. 实修

\origpageno=39
\pageno=37

{
\parindent=3pc
\noindent\hang\hangafter-2
\1\smash{\lower19pt\hbox to 0pt{\hskip-\hangindent\KT{36}虽\hfill}}%
{\KT{12}然禅修的科目多种多样},但我们强烈建议你从不间断地专注于你的呼
吸以获得一定程度的基本的定\myfootnote{定: concentration,专注。}开始。在
做这个时请记住,你不是在修炼深度禅定\myfootnote{禅定: absorption。}或
单纯的专注技巧。你是在修炼正念,因此你只需要一定程度的基本的定。你想要
的是培养正念以最终产生证悟如实真理的参悟和智慧。你想要的是如实认知身心
综合体的运作机理。你想要的是消除所有心理烦恼使你的人生真正平静和快乐。

}

没有看到事物的真相内心就无法清净。“如实知见”是如此满含深意又含糊不清的
短语。许多初学者不明白我们所说的意思,因为看似任何具有清楚视力的人应该
都可以如实地看清事物。

当我们用这个短语来指代从禅修中获得的参悟时,我们并不是指通过普通的眼睛
从表面看清事物,而是指通过智慧从它们的内在如实地明白它们。用智慧知见意
指不带着由贪、瞋、痴所产生成见和偏颇从我们身心综合体的框架内看清事物。


\endchapter

\byebye
