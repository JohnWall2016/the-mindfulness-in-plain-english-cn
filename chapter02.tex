% -*- coding: utf-8 -*-

\input macros

\beginchapter Chapter 2. 何非修禅

\origpageno=11
\pageno=11

{
\parindent=3pc
\noindent\hang\hangafter-2
\1\smash{\lower19pt\hbox to 0pt{\hskip-\hangindent\KT{36}禅\hfill}}%
{\KT{12}修是一个词语}。你之前听过这个词语,否则你不会选择这本书。思维的
过程以联想的方式运作,而各种各样的观点与“禅修”这个词相关联。有一些可能
是准确的,而其它一些则可能是胡说八道。另外一些更适合其它的禅修体系而与
内观\myfootnote{vipassana:insight,内观,洞见。}禅修没有任何关系。在继
续讲解之前,我们有必要清除头脑中的一些成见以使新的信息能无碍地通过。让
我们从一些最明显的内容开始。

}

我们不会教你随观你的肚脐或吟唱神秘的音节。你不会要战胜恶魔或是控制无形
的力量。不会因你的成绩而授予彩色的腰带,你也不需要剔光头或戴上包头巾。
你甚至不需要散尽你的财产住进寺院。事实上,除非你的生活不道德或混乱不堪,
你很可能马上开始而且有所斩获。听上去鼓舞人心,难道不是吗?

有很多关于禅修主题的书。很多这些书是从一个直接存在于某个特定的宗教或哲
学传统中的观点来著述的,而且很多这些作者都没有在文中指出这一点。他们关
于禅修的论述听上去像是一般的规律,而实际上却是专属于那个特定实修系统的
非常\1特有的步骤。更严重的是,这些大量存在的复杂理论和解释往往是相互矛盾
的。结果真是一团糟:大量无关的数据伴随着庞大的混乱的相互冲突的观点。本
书的内容是特定的。我们专门讲解的是内观禅修系统。我们将教你以一种平静而
超脱的方式观察你自己内心的运行,以使你可以获得对你自己行为的洞见。其目
标是觉知,一种非常强烈、专注和精调的觉知,以使你能够穿透实在自身的内部
运作。

关于禅修存在一些普遍的误解。我们看到同样的问题在新的学生身上反复出现。
我们最好一次性解决好这些问题,因为它们是那种一开始就会阻止你前行的成见。
我们将逐一讨论并一一化解它们。

\subsectnon 误解1:禅修仅是一种放松的技巧。

这里令人不安的是{\it 仅是}这个词。放松是禅修的一个关键部分,但内观禅修
着眼更高的目标。这个论断对于其它很多禅修系统来说是基本属实的。所有的禅修
过程都强调内心的专注,使内心安注于一个项目或一个思维区域。当专注足够强烈
和全然时,你将获得深度且喜悦的放松,我们称之为{\it 禅定\myfootnote{jhana:禅定,禅那。}}。
它是如此极度平静的状态近乎狂喜,一种高于且超越任何正常的意识状态可以体
验到的事物的快乐形式。多数禅修系统只停留于此。禅定是其目标,而当你达到它后,
你只是终其余生重复这个经验。内观禅修\1并非如此。内观寻求另一个目的:觉知。
专注和放松被视为觉知必需的伴随物。它们是必要的前驱、应手的工具和有益的
副产品。但它们不是目的。目的是洞见。内观禅修是一种深远的宗教修行,它不为另
的只为你日常生活的净化和转化。我们将在第14章中更彻底地探讨定与观的区别。

\endchapter

\vfill\eject\byebye
