% -*- coding: utf-8 -*-

\input macros

\beginchapter Chapter 2. 何非修禅

\origpageno=11
\pageno=11

{
\parindent=3pc
\noindent\hang\hangafter-2
\1\smash{\lower19pt\hbox to 0pt{\hskip-\hangindent\KT{36}禅\hfill}}%
{\KT{12}修是一个词语}。你之前听过这个词语,否则你不会选择这本书。思维的
过程以联想的方式运作,而各种各样的观点与“禅修”这个词相关联。有一些可能
是准确的,而其它一些则可能是胡说八道。另外一些更适合其它的禅修体系而与
内观\myfootnote{vipassana:insight,内观,洞见。}禅修没有任何关系。在继
续讲解之前,我们有必要清除头脑中的一些成见以使新的信息能无碍地通过。让
我们从一些最明显的内容开始。

}

我们不会教你随观你的肚脐或吟唱神秘的音节。你不会要战胜恶魔或是控制无形
的力量。不会因你的成绩而授予彩色的腰带,你也不需要剔光头或戴上包头巾。
你甚至不需要散尽你的财产住进寺院。事实上,除非你的生活不道德或混乱不堪,
你很可能马上开始而且有所斩获。听上去鼓舞人心,难道不是吗?

有很多关于禅修主题的书。很多这些书是从一个直接存在于某个特定的宗教或哲
学传统中的观点来著述的,而且很多这些作者都没有在文中指出这一点。他们关
于禅修的论述听上去像是一般的规律,而实际上却是专属于那个特定实修系统的
非常\1特有的步骤。更严重的是,这些大量存在的复杂理论和解释往往是相互矛盾
的。结果真是一团糟:大量无关的数据伴随着庞大的混乱的相互冲突的观点。本
书的内容是特定的。我们专门讲解的是内观禅修系统。我们将教你以一种平静而
超脱的方式观察你自己内心的运行,以使你可以获得对你自己行为的洞见。其目
标是觉知,一种非常强烈、专注和精调的觉知,以使你能够穿透实在自身的内部
运作。

关于禅修存在一些普遍的误解。我们看到同样的问题在新的学生身上反复出现。
我们最好一次性解决好这些问题,因为它们是那种一开始就会阻止你前行的成见。
我们将逐一讨论并一一化解它们。

\subsectnon 误解1:禅修仅是一种放松的技巧。

这里令人不安的是{\it 仅是}这个词。放松是禅修的一个关键部分,但内观禅修
着眼更高的目标。这个论断对于其它很多禅修系统来说是基本属实的。所有的禅修
过程都强调内心的专注,使内心安注于一个项目或一个思维区域。当专注足够强烈
和全然时,你将获得深度且喜悦的放松,我们称之为{\it 禅定\myfootnote{jhana:禅定,禅那。}}。
它是如此极度平静的状态近乎狂喜,一种高于且超越任何正常的意识状态可以体
验到的事物的快乐形式。多数禅修系统只停留于此。禅定是其目标,而当你达到它后,
你只是终其余生重复这个经验。内观禅修\1并非如此。内观寻求另一个目的:觉知。
专注和放松被视为觉知必需的伴随物。它们是必要的前驱、应手的工具和有益的
副产品。但它们不是目的。目的是洞见。内观禅修是一种深远的宗教修行,它不为另
的只为你日常生活的净化和转化。我们将在第14章中更彻底地探讨定与观的区别。

\subsectnon 误解2:禅修意味着进入催眠状态。

这个论断仍只是对某些禅修系统的准确描述,但无法应用于内观禅修。内观禅修不
是一种催眠的形式。你不会试图使你内心昏迷变得无意识,或将自己变成无情感的
植物人。如果有所不同的话,那就是情况恰恰相反:你变得越来越熟悉自己的情感
变化。你将学会更清晰和准确地认识你自己。在学习这个技巧的过程中,某些状态
在观察者看来确实像被催眠似的。但实际上完全相反。在精神催眠中,主体易受他
人控制,而在深度禅定之中,禅修者仍保留了大量的自我控制。这个相似性只是表
面的,而且在任何情况下,这些现象的出现都不是内观禅修的重点。就像我们所说
的,深度禅定只是通向更高觉知的工具或垫脚石。内观依其定义是正念或觉知的修
炼。如果你发现在禅修中变得无意识,则依内观系统中对内观这个词的定义,你其
没有在禅修。

\subsectnon \1误解3:禅修是无法理解的神秘修行。

这种说法基本是对的,但不全然。禅修处理比概念思维更深的意识层次。因此,一
些禅修的经验无法用语言表达。但这并不意味着禅修无法理解。有比使用语言更深
的理解事物的方式。你能理解如何行走。但你可能无法描述在这个过程中你的神经
纤维和肌肉收缩的准确顺序。但你知道如何做它。禅修需要以同样的方式理解——通
过做它。它不是你可以用抽象的术语或用语言谈论的事物。它是被经验的事物。禅
修不是给出自动和可预见结果的无需用心的公式;你从来无法真正准确预见在任何
一次特定的修炼的过程中将会出现什么。每一次它都是一次探究和一次实证、一次
冒险。事实上,这种说法是非常准确的,当你在实修中达到可预见性和千篇一律的
感受时,你可以视其为你已经偏离正道和走向停滞的标志。学会视每一秒为宇宙中
的第一和唯一的一秒是内观禅修的基本功。

\subsectnon 误解4:禅修的目的是使人具有超常的精神力量。

不是。禅修的目的是培养觉知。学会读心术不是重点。悬浮空中不是目标。目标是
解脱。超常精神力量现象和禅修之间存在某种关联,但其关系很复杂。在禅修者修
行的初级阶段,这些现象有可能会也有可能不会出现。一些人可能会\1经验到一些
直觉的理解或前世的记忆;而另一些人则不会。在任何情况下,这些现象都不应视
为充分发展的和可靠的精神力量,而且它们不应给予过度的强调。这些现象事实上
对于禅修初学者来说是非常危险的,因为它们太有诱惑性了。它们可以成为自我的
陷阱,引诱你脱离正轨。你最好的处理方式是不去强调这些现象。如果它们出现了,
那很好。如果它们没有出现,那也很好。在禅修者的生涯中有这么一个时期他或者
她会进行特殊的修练以培养超常的精神力量。但这发生在禅修很长一段时间之后。
只有在禅修者已达到很深的禅定阶段之后,他或她才足够成熟来修练这些能力而不
会有让它们失控或控制他或她生命的危险。时机到来时禅修者将专门为了服务他人
的目的而开发它们。在多数情况下,这种状况只有在几十年的修行之后才会发生。
不要为此而担心。只需要专注于培养越来越多的觉知。如果声音和幻象出现时,只
需注意它们然后顺其自然。不要牵涉其中。

\subsectnon 误解5:禅修有危险,谨慎的人应避免它。

任何事情都有危险。走在街上你可能被巴士撞到。沐浴时你可能扭断脖子。修禅,
而你可能从你的过去翻出各种各样令你不悦的事物。埋藏很久的被压抑的事物可能
是很恐怖的。但探索它也能令人受益匪浅。没有什么活动是完全没有风险的,但这
并不意味着你应该将自己包裹在保护茧中。那不是生活,而是提前死亡。应对危险
的方式应是知道它大致存在多少,哪儿可能遇见它,以及当它出现后\1怎么处理它。
这正是这本指南的目的。内观禅修是对觉知的开发。它本身并不危险;相反,提升
的觉知是使我们远离危险的守护者。修练得当的话,禅修是非常柔和和渐进的过程。
慢慢地放松地进行,你实修的进展将非常自然地发生。不要强迫任何事情。之后,
在一位胜任的师父的密切关注和守护的智慧之下,你可以采取一段时间的密集禅修
来提高你成长的速度。但在开始时,慢慢来。柔和地练习就不会出问题。

\subsectnon 误解6:禅修是为圣人和苦行僧,而非为普通人准备的。

这个态度在亚洲国家是非常普遍的,在这些国家僧人和圣人被给予了大量的仪式上
的礼敬,有点像美国人崇拜电影明星和棒球英雄的态度。这些人被定型了,他们的
人生被放大,而且被付予各种各样几乎无人能做到的品质。即使在西方,我们对禅
修也有相似的观点。我们以为禅修者是极度虔诚的大人物,在他们嘴里黄油都不敢
溶化。与这些人的一点点的亲身接触将很快驱散这种幻觉。他们往往被发现是充满
精力和激情的人,他们在生活中体现出惊人的活力。

当然,那些最神圣的人修禅,但是他们修禅不是因为他们是圣人。情况恰恰相反。
他们是圣人正因为他们修禅;禅修是他们成为圣人的原因。他们在成为圣人之前就
已开始禅修,否则他们不会成圣。这是一个重要的观点。相当多的初学者似乎觉得
一个人\1在开始禅修之前应是道德高尚的。这是一个不切实际的策略。德行需要一定
程度的心理控制力作为前题条件。如果没有一点自控力你将无法遵循任何道德准则,
而且如果你的内心像老虎机的水果图形柱一样不停旋转地话,自控力是很难获得的。
所以应优先心理的培养。

在佛教的禅修中存在三个互为一体的要素——戒、定、慧\myfootnote{morality,
concentration, and wisdom:戒、定、慧;德行、禅定、智慧。}。这三个要素在我
们修行深入后会共同成长。它们彼此相互影响,所以你要同时培养这三个要素,不要
将其分开。当你具有真正理解某个事态的智慧时,对于介入之中各方的慈悲将自动
发生,而慈悲意味着你会自动约束自己以避免任何伤害自己或他人的思想、言语和行
动;因而,你的行为自然而然是合乎道德的。只有在你没有深入地理解事物时,你才
会造成问题。在你不能看清你行动的后果时,你将犯下无知的错误。等待自己道德高
尚后再开始修禅的人是在痴心妄想。古代的智者说这种人是在等待大海平静后再下海
洗澡。

为了全面地理解这个关系,让我们用德行存在的不同层次来解释。最底层的德行是坚
守某个人订立的一系列规则和条例。它可以是你拥护的先知。它可能是国家、宗族的
首领、或是父母。无论是谁创立的这些规则,在这个层次上你所要做的只是知道这些
规则并遵守它们。机器人都可以这样做。甚至是训练有素的猩猩也可以做到,只要这
些规则足够简单并且在它犯规时被木棍惩罚。这个层次根本不需要禅修。你所需要的
只是规则和挥舞木棍的人。

下一个层次的德行是即使\1没有惩罚你的人在场时你仍能遵守相同的规则。你遵守是
因为你已经将这些规则内部化。每次你违反它时你会自我处罚。这个层次需要一点自
控力。但是如果你的思维模式是混乱的,你的行为也将是混乱的。心理的修养将减少
内心的混乱。

第三个层次的德行最好称其为“伦理”。这个层次是对前两个层次的飞跃,是方向上的
完全改变。在这个伦理的层面上,一个人不再默守权威施加的规则。他会选择遵循正
念、智慧和慈悲所指明的道路。这个层次需要真正的才智,和在任何事态下平衡各种
因素的能力,借以做出独特的、创造性的和适当的反应。并且,做出这些决定的个人
必须使他或她自己走出狭隘的个人观点。这个人必须从一个客观的观点来看待整个事
态,给予他或她自己的需求和其他的需求以平等的权衡。换句话说,他或她必须摆脱
贪婪、仇恨、忌妒,以及其它一切通常使我们无法看到他人面对的问题的自私的东西。
只有这个时候他或她才能选出对于事态真正优化的一系列精准行为。这个层次的德行
绝对需要禅修,除非你生下来就是圣人。没有其它的方式可以获得这个技巧。加之,
在这个层次所需的处世之道使人疲惫不堪。如果你试图在每个情况下都用你刻意的思
维来平衡所有因素的话,你将使自己超出负荷。智商无法做到同时平衡这么多的事物。
庆幸的是,一种更深层次的觉知可以轻松完成这类处理。禅修能帮你达成这种处世之
道。它是一种很奇特的感受。

有一天你遇到某个问题——比如说,处理赫曼叔叔最近的离婚事件。它看似完全无解,
一大堆充满了“可能”的混乱场面\1甚至会令善于处理矛盾的所罗门王\myfootnote{%
King Solomon:所罗门王}都头痛。隔天你正在洗盘子,头脑里想的完全是其它事情,
但是突然你对赫曼叔叔的离婚事件豁然开朗。解决方案从你的内心深处跳出来,然后
你说,“啊哈!”接着整个事件就此解决。只有在你将要解决的问题与你的逻辑思维相
脱离,并给你内心深处以机会来酝酿解决方案时,这种直觉才能出现。刻意的思维只
会阻碍事情的解决。禅修将教你如何使你自己与思维的过程相脱离。它是走出你自身
困境的心理艺术,而且也是日常生活中非常有用的技巧。禅修显然不是只对苦行僧和
隐士有用与我们则不相关的修练。它是聚焦于日常事件而且在每个人的生活中都有直
接应用的实用技巧。禅修不是“超凡脱俗的”。

不幸的是,对于某些新修者来说正是这个事实使他们不满。他们走入禅修是为了即刻
的宇宙启示,并伴随着天使的合奏。而他们往往得到的是更高效的处理垃圾的方式以
及应对赫曼叔叔的更好的方法。他们没有必要地失望。垃圾处理方案先出现。大天使
的声音则需要更长的时间。

\subsectnon 误解7:禅修是逃避现实。

不正确。禅修是直接投入现实之中。它不会使你与人生的疼痛相互隔绝,而是让你
深入地探寻人生和它的方方面面,以使你穿透疼痛的壁垒并超越痛苦。内观禅修是
以特定的面对现实的意向来练习的实修方式,它是以其所是的方式来体验人生并如
实地应对你所发现的一切。它让你拨开幻象并使你摆脱一直以来你对自己所说的官
冕堂皇的谎言。是什么就是什么。你就是你,而你用谎言来掩饰自己的\1弱点和动机
只会让你更加无法摆脱它们。内观禅修不是试图让你忘记自己或是掩盖你的问题。
它是让你学会如实地看待自己并且全然地接受它。只有这样你才能改变它。

\subsectnon 误解8:禅修是让人兴奋的好方法。

嗯,既对也不对。禅修有时确实会产生美妙的极乐感受。但这些感受并不是目的,
而且它们也不总是出现。甚至,如果你心里想着这个目的来禅修,这些感受将比为
了禅修的真正目的,即增强的觉知,来禅修更难以出现。极乐的结果来自于放松,
而放松的结果来自于紧张的释放。从禅修中寻求喜乐会将紧张引入这个过程中,从
而破坏整个事件链。这是一个矛盾:你唯有不追逐喜乐方可体验到喜乐。喜乐感不
是禅修的目的。它可能经常出现,但应视其为副产品。但它不失为一种非常快乐的
副作用,而且随着你禅修时间的增长它变得越来越频繁。这个你可以从老修行的口
中得到证实。

\subsectnon 误解9:禅修是自私的行为。

它看上去确实如此。禅修者悠然地坐在一个小垫子上。她有去献血吗?没有。她有
去积极地救援灾难受害者吗?没有。但让我们来调查一下她的动机。她为何这样做?
禅修者的目的是为了清除自己心里的愤怒、偏见和恶意,并且她积极地投入到消除
贪欲、紧张和麻木的过程中。\1正是这些心理阻碍了她对别人的同情。除非这些心理
被去除,否则她做的任何善事可能只是自我意识的扩张,从长期来看并非真正有所
帮助。以帮助之名行伤害之实历来就有。西班牙宗教法庭的大审判长\myfootnote{%
The grand inquisitor of Spanish Inquisition}鼓吹最崇高的目的。赛勒姆巫师
审判\myfootnote{The Salem witchcraft trails}的执行是为了“公益”。通过调查
资深禅修者的个人生活,你常常会发现他们积极参与人道主义事业。你很少发现他
们像圣战使团一样为了自以为是的虔诚理念而情愿牺牲特定的个体。事实是我们比
自己所知道的更加自私。如果自我被放任自由,它将有方法将崇高的活动变得一文
不值。通过禅修,借由对我们表现自私的很多微妙方式的醒悟,我们变得对自己如
实地觉知。从而我们开始成为真正的无私。洗涤你的自私并不是自私的行为。

\subsectnon 误解10:在你修禅时,你闲坐着思考崇高的思想。

这种说法也是错的。有某些深思的体系是在做这种事。但不是内观禅修。内观是觉
知的修练,对当下事物的觉知,无论它是崇高的真理还是不值一提的垃圾。

\endchapter

\byebye
