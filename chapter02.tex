% -*- coding: utf-8 -*-

\input macros

\beginchapter Chapter 2. 何非修禅

\origpageno=11
\pageno=11

{
\parindent=3pc
\noindent\hang\hangafter-2
\1\smash{\lower19pt\hbox to 0pt{\hskip-\hangindent\KT{36}禅\hfill}}%
{\KT{12}修是一个词语}。你之前听过这个词语,否则你不会选择这本书。思维的
过程以联想的方式运作,而各种各样的观点与“禅修”这个词相关联。有一些可能
是准确的,而其它一些则可能是胡说八道。另外一些更适合其它的冥想体系而与
内观禅修没有任何关系。在继续讲解之前,我们有必要清除头脑中的一些成见以
使新的信息能无碍地通过。让我们从一些最明显的内容开始。

}

我们不会教你随观你的肚脐或吟唱神秘的音节。你不会要战胜恶魔或是控制无形
的力量。不会因你的成绩而授予彩色的腰带,你也不需要剔光头或戴上包头巾。
你甚至不需要散尽你的财产住进寺院。事实上,除非你的生活不道德或混乱不堪,
你很可能马上开始而且有所斩获。听上去鼓舞人心,难道不是吗?



\endchapter

\vfill\eject\byebye
