% -*- coding: utf-8 -*-

\input macros

\beginchapter Chapter 2. 何非修禅

\origpageno=11
\pageno=11

{
\parindent=3pc
\noindent\hang\hangafter-2
\1\smash{\lower19pt\hbox to 0pt{\hskip-\hangindent\KT{36}禅\hfill}}%
{\KT{12}修是一个词语}。你之前听过这个词语,否则你不会选择这本书。思维的
过程以联想的方式运作,而各种各样的观点与“禅修”这个词相关联。有一些可能
是准确的,而其它一些则可能是胡说八道。另外一些更适合其它的禅修体系而与
内观\myfootnote{vipassana:insight,内观,洞见。}禅修没有任何关系。在继
续讲解之前,我们有必要清除头脑中的一些成见以使新的信息能无碍地通过。让
我们从一些最明显的内容开始。

}

我们不会教你随观你的肚脐或吟唱神秘的音节。你不会要战胜恶魔或是控制无形
的力量。不会因你的成绩而授予彩色的腰带,你也不需要剔光头或戴上包头巾。
你甚至不需要散尽你的财产住进寺院。事实上,除非你的生活不道德或混乱不堪,
你很可能马上开始而且有所斩获。听上去鼓舞人心,难道不是吗?

有很多关于禅修主题的书。很多这些书是从一个直接存在于某个特定的宗教或哲
学传统中的观点来著述的,而且很多这些作者都没有在文中指出这一点。他们关
于禅修的论述听上去像是一般的规律,而实际上却是专属于那个特定实修系统的
非常\1特有的步骤。更严重的是,这些大量存在的复杂理论和解释往往是相互矛盾
的。结果真是一团糟:大量无关的数据伴随着庞大的混乱的相互冲突的观点。本
书的内容是特定的。我们专门讲解的是内观禅修系统。我们将教你以一种平静而
超脱的方式观察你自己内心的运行,以使你可以获得对你自己行为的洞见。其目
标是觉知,一种非常强烈、专注和精调的觉知,以使你能够穿透实在自身的内部
运作。

关于禅修存在一些普遍的误解。我们看到同样的问题在新的学生身上反复出现。
我们最好一次性解决好这些问题,因为它们是那种一开始就会阻止你前行的成见。
我们将逐一讨论并一一化解它们。

\subsectnon 误解1:禅修仅是一种放松的技巧。

这里令人不安的是{\it 仅是}这个词。放松是禅修的一个关键部分,但内观禅修
着眼更高的目标。这个论断对于其它很多禅修系统来说是基本属实的。所有的禅修
过程都强调内心的专注,使内心安注于一个项目或一个思维区域。当专注足够强烈
和全然时,你将获得深度且喜悦的放松,我们称之为{\it 禅定\myfootnote{jhana:禅定,禅那。}}。
它是如此极度平静的状态近乎狂喜,一种高于且超越任何正常的意识状态可以体
验到的事物的快乐形式。多数禅修系统只停留于此。禅定是其目标,而当你达到它后,
你只是终其余生重复这个经验。内观禅修\1并非如此。内观寻求另一个目的:觉知。
专注和放松被视为觉知必需的伴随物。它们是必要的前驱、应手的工具和有益的
副产品。但它们不是目的。目的是洞见。内观禅修是一种深远的宗教修行,它不为另
的只为你日常生活的净化和转化。我们将在第14章中更彻底地探讨定与观的区别。

\subsectnon 误解2:禅修意味着进入催眠状态。

这个论断仍只是对某些禅修系统的准确描述,但无法应用于内观禅修。内观禅修不
是一种催眠的形式。你不会试图使你内心昏迷变得无意识,或将自己变成无情感的
植物人。如果有所不同的话,那就是情况恰恰相反:你变得越来越熟悉自己的情感
变化。你将学会更清晰和准确地认识你自己。在学习这个技巧的过程中,某些状态
在观察者看来确实像被催眠似的。但实际上完全相反。在精神催眠中,主体易受他
人控制,而在深度禅定之中,禅修者仍保留了大量的自我控制。这个相似性只是表
面的,而且在任何情况下,这些现象的出现都不是内观禅修的重点。就像我们所说
的,深度禅定只是通向更高觉知的工具或垫脚石。内观依其定义是正念或觉知的修
炼。如果你发现在禅修中变得无意识,则依内观系统中对内观这个词的定义,你其
没有在禅修。

\subsectnon \1误解3:禅修是无法理解的神秘修行。

这种说法基本是对的,但不全然。禅修处理比概念思维更深的意识层次。因此,一
些禅修的经验无法用语言表达。但这并不意味着禅修无法理解。有比使用语言更深
的理解事物的方式。你能理解如何行走。但你可能无法描述在这个过程中你的神经
纤维和肌肉收缩的准确顺序。但你知道如何做它。禅修需要以同样的方式理解——通
过做它。它不是你可以用抽象的术语或用语言谈论的事物。它是被经验的事物。禅
修不是给出自动和可预见结果的无需用心的公式;你从来无法真正准确预见在任何
一次特定的修炼的过程中将会出现什么。每一次它都是一次探究和一次实证、一次
冒险。事实上,这种说法是非常准确的,当你在实修中达到可预见性和千篇一律的
感受时,你可以视其为你已经偏离正道和走向停滞的标志。学会视每一秒为宇宙中
的第一和唯一的一秒是内观禅修的基本功。

\subsectnon 误解4:禅修的目的是使人具有超常的精神力量。

不是。禅修的目的是培养觉知。学会读心术不是重点。悬浮空中不是目标。目标是
解脱。超常精神力量现象和禅修之间存在某种关联,但其关系很复杂。在禅修者修
行的初级阶段,这些现象有可能会也有可能不会出现。一些人可能会\1经验到一些
直觉的理解或前世的记忆;而另一些人则不会。在任何情况下,这些现象都不应视
为充分发展的和可靠的精神力量,而且它们不应给予过度的强调。这些现象事实上
对于禅修初学者来说是非常危险的,因为它们太有诱惑性了。它们可以成为自我的
陷阱,引诱你脱离正轨。你最好的处理方式是不去强调这些现象。如果它们出现了,
那很好。如果它们没有出现,那也很好。在禅修者的生涯中有这么一个时期他或者
她会进行特殊的修练以培养超常的精神力量。但这发生在禅修很长一段时间之后。
只有在禅修者已达到很深的禅定阶段之后,他或她才足够成熟来修练这些能力而不
会有让它们失控或控制他或她生命的危险。时机到来时禅修者将专门为了服务他人
的目的而开发它们。在多数情况下,这种状况只有在几十年的修行之后才会发生。
不要为此而担心。只需要专注于培养越来越多的觉知。如果声音和幻象出现时,只
需注意它们然后顺其自然。不要牵涉其中。

\subsectnon 误解5:禅修有危险,谨慎的人应避免它。

任何事情都有危险。走在街上你可能被巴士撞到。沐浴时你可能扭断脖子。修禅,
而你可能从你的过去翻出各种各样令你不悦的事物。埋藏很久的被压抑的事物可能
是很恐怖的。但探索它也能令人受益匪浅。没有什么活动是完全没有风险的,但这
并不意味着你应该将自己包裹在保护茧中。那不是生活,而是提前死亡。应对危险
的方式应是知道它大致存在多少,哪儿可能遇见它,以及当它出现后\1怎么处理它。
这正是这本指南的目的。内观禅修是对觉知的开发。它本身并不危险;相反,提升
的觉知是使我们远离危险的守护者。修练得当的话,禅修是非常柔和和渐进的过程。
慢慢地放松地进行,你实修的进展将非常自然地发生。不要强迫任何事情。之后,
在一位胜任的师父的密切关注和守护的智慧之下,你可以采取一段时间的密集禅修
来提高你成长的速度。但在开始时,慢慢来。柔和地练习就不会出问题。

\subsectnon 误解6:禅修是为圣人和高僧,而非为普通人准备的。

这个态度在亚洲国家是非常普遍的,在这些国家僧人和圣人被给予了大量的仪式上
的礼敬,有点像美国人崇拜电影明星和棒球英雄的态度。这些人被定型了,他们的
人生被放大,而且被付予各种各样几乎无人能做到的品质。即使在西方,我们对禅
修也有相似的观点。我们以为禅修者是极度虔诚的大人物,在他们嘴里黄油都不敢
溶化。与这些人的一点点的亲身接触将很快驱散这种幻觉。他们往往被发现是充满
精力和激情的人,他们在生活中体现出惊人的活力。

当然,那些最神圣的人修禅,但是他们修禅不是因为他们是圣人。情况恰恰相反。
他们是圣人正因为他们修禅;禅修是他们成为圣人的原因。他们在成为圣人之前就
已开始禅修,否则他们不会成圣。这是一个重要的观点。相当多的初学者似乎觉得
一个人在开始禅修之前应是道德高尚的。这是一个不切实际的策略。德行需要一定
程度的心理控制作为前题条件。

\endchapter

\byebye
