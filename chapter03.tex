% -*- coding: utf-8 -*-

\input macros

\beginchapter Chapter 3. 何为修禅

\origpageno=23
\pageno=21

{
\parindent=3pc
\noindent\hang\hangafter-2
\1\smash{\lower19pt\hbox to 0pt{\hskip-\hangindent\KT{36}禅\hfill}}%
{\KT{12}修是一个词语},而词语对于不同说话的人有不同的用法。这个看似是稀松
平常的观点,但实际却非如此。准确地区分特定说话的人所用词语的含义是非常
重要的。几乎地球上每一个文明都曾创造了某种可称为禅修的心理修行。这完全取
决于你给予禅修这个词多么宽泛的定义。世界上称为禅修的技巧千奇百怪,我们不
会在本书中逐一调查它们。自有其它的书做这样的事。为了本书的目的,我们只会
讨论西方读者最熟悉且与禅修这个术语关系最密切的那些修行。

}

在犹太基督教\myfootnote{Judeo-Christian}的传统中,我们发现两个相互重叠的修
行,它们被称为祷告和沉思\myfootnote{Prayer and Contemplation}。祷告是直接向
某个神灵的告白。沉思是长时间的对一个特定主题的有意识的思考,这个主题一般
是一个宗教的理想或是一段经文。从心理培养的立场来看,这些活动都是在专注上的
练习。平常的意识思想的洪流被限制住,而内心被代入到一个意识的运作范围之内。
其结果就像你在任何的专注练习中发现的一样:深度的安定,新陈代谢的生理上的放
缓,以及平静和安乐的感觉。

从印度教传统中产生的瑜伽禅修\myfootnote{Yogic meditation}也是\1纯粹的专注
的练习。这些传统的基础练习包括将内心聚焦于一个单独的对象上——一块石头、烛光、
音节或者其它的东西——并且不让内心走神。获得这个基本的技巧后,瑜伽师\myfootnote{%
yogi}通过采取更复杂的禅修对象——颂歌、彩色的宗教图像、身体上的能量通道,等
等——来拓展他的实修。但是,无论禅修的对象多么复杂,瑜伽禅修本身仍只是单纯的
专注的练习。

在佛教的传统中,专注(定)也被赋予了很高的价值。但一个新的元素被加入并且给
予了更高的强调:觉知的元素。所有佛教的禅修都以发展觉知为目的,而用专注作为
达到这个目的的工具。但是佛教的传统是非常广泛的,存在几条不同的通向这个目标
的途径。禅宗的禅修\myfootnote{Zen meditation:禅宗的禅修。一种是顿悟,另一
种是参话头。}使用两种独立的方法。第一种是用纯粹的意志力直接投入到觉知中。
你打坐就只是打坐,意味着抛开你头脑中的一切除了对打坐的纯粹觉知。这听上去非
常简单。但事实并非如此。(简单的测试一下就会发现它实际上是多么的困难。)第
二种方式,为日本临济宗\myfootnote{Rinzai school}所使用,是欺骗内心使其走出
意识思维并进入到纯粹的觉知中。它是通过交给学生一个无解的迷题使他必须解决,
从而将其置于极度痛苦的训练状态。因为他无法从这个状态的痛苦中逃脱,他必须遁
入对当下的纯粹体验中:没有其它的地方可去。禅宗是非常艰苦的。它对很多人有效,
但是它真的非常艰苦。

\endchapter

\byebye
