% -*- coding: utf-8 -*-

\input macros

\beginchapter Chapter 3. 何为修禅

\origpageno=23
\pageno=21

{
\parindent=3pc
\noindent\hang\hangafter-2
\1\smash{\lower19pt\hbox to 0pt{\hskip-\hangindent\KT{36}禅\hfill}}%
{\KT{12}修是一个词语},而词语对于不同的说话人有不同的用法。这个看似是稀松
平常的观点,但实际却非如此。准确地区分特定说话的人所用词语的含义是非常
重要的。几乎地球上每一个文明都曾创造了某种可称为禅修的心理修行。这完全取
决于你给予禅修这个词多么宽泛的定义。世界上称为禅修的技巧千奇百怪,我们不
会在本书中逐一调查它们。自有其它的书做这样的事。为了本书的目的,我们只会
讨论西方读者最熟悉且与禅修这个术语关系最密切的那些修行。

}

在犹太基督教\myfootnote{Judeo-Christian}的传统中,我们发现两个相互重叠的修
行,它们被称为祷告和沉思\myfootnote{Prayer and Contemplation}。祷告是直接向
某个神灵的告白。沉思是长时间的对一个特定主题的有意识的思考,这个主题一般
是一个宗教的理想或是一段经文。


\endchapter

\byebye
