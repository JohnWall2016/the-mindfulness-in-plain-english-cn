% -*- coding: utf-8 -*-

\input macros

\beginchapter Chapter 3. 何为修禅

\origpageno=23
\pageno=21

{
\parindent=3pc
\noindent\hang\hangafter-2
\1\smash{\lower19pt\hbox to 0pt{\hskip-\hangindent\KT{36}禅\hfill}}%
{\KT{12}修是一个词语},而词语对于不同说话的人有不同的用法。这个看似是稀松
平常的观点,但实际却非如此。准确地区分特定说话的人所用词语的含义是非常
重要的。几乎地球上每一个文明都曾创造了某种可称为禅修的心理修行。这完全取
决于你给予禅修这个词多么宽泛的定义。世界上称为禅修的技巧千奇百怪,我们不
会在本书中逐一调查它们。自有其它的书做这样的事。为了本书的目的,我们只会
讨论西方读者最熟悉且与禅修这个术语关系最密切的那些修行。

}

在犹太基督教\myfootnote{Judeo-Christian}的传统中,我们发现两个相互重叠的修
行,它们被称为祷告和沉思\myfootnote{Prayer and Contemplation}。祷告是直接向
某个神灵的告白。沉思是长时间的对一个特定主题的有意识的思考,这个主题一般
是一个宗教的理想或是一段经文。从心理培养的立场来看,这些活动都是在专注上的
练习。平常的意识思想的洪流被限制住,而内心被代入到一个意识的运作范围之内。
其结果就像你在任何的专注练习中发现的一样:深度的安定,新陈代谢的生理上的放
缓,以及平静和安乐的感觉。

从印度教传统中产生的瑜伽禅修\myfootnote{Yogic meditation}也是\1纯粹的专注
的练习。这些传统的基础练习包括将内心聚焦于一个单独的对象上——一块石头、烛光、
音节或者其它的东西——并且不让内心走神。获得这个基本的技巧后,瑜伽师\myfootnote{%
yogi}通过采取更复杂的禅修对象——颂歌、彩色的宗教图像、身体上的能量通道,等
等——来拓展他的实修。但是,无论禅修的对象多么复杂,瑜伽禅修本身仍只是单纯的
专注的练习。

在佛教的传统中,专注(定)也被赋予了很高的价值。但一个新的元素被加入并且给
予了更高的强调:觉知的元素。所有佛教的禅修都以发展觉知为目的,而用专注作为
达到这个目的的工具。但是佛教的传统是非常广泛的,存在几条不同的通向这个目标
的途径。禅宗的禅修\myfootnote{Zen meditation:禅宗的禅修。一种是顿悟,另一
种是参话头。}使用两种独立的方法。第一种是用纯粹的意志力直接投入到觉知中。
你打坐就只是打坐,意味着抛开你头脑中的一切除了对打坐的纯粹觉知。这听上去非
常简单。但事实并非如此。(简单的测试一下就会发现它实际上是多么的困难。)第
二种方式,为日本临济宗\myfootnote{Rinzai school}所使用,是欺骗内心使其走出
意识思维并进入到纯粹的觉知中。它是通过交给学生一个无解的迷题使他必须解决,
从而将其置于极度痛苦的训练状态。因为他无法从这个状态的痛苦中逃脱,他必须遁
入对当下的纯粹体验中:没有其它的地方可去。禅宗是难以做到的。它对很多人有
效,但是它真的非常难做到。

另一个技巧,密宗佛教\myfootnote{Tantric Buddhism},则几乎相反。意识思维,至少
在我们平常的运作方式下,是我执\myfootnote{ego: 自我,我执,对我的执著。}的表
现,“你”往往被你自认为是你自己。意识思维与自我的观念紧紧相连。自我意识或我
执只是一系列的反应和心理印象,它们被人为地附着在纯然觉知的流动过程中。\1密
宗寻求通过摧毁我执的形象来获得纯然的觉知。它是通过一个观想的过程来达成的。
学生被给予一个特定的宗教形象来冥想,比如,某位密宗信奉的大士。她以非常彻底
的方式冥想以至于她自己变成了那位大士。她脱下她自认的身份而套上另一个。这需
要一定的时间,如你所想的一样,但它会起作用。在这个过程中,她能观察到我执被
构造并实现的方式。她开始认识到所有我执的非理性,包括她自己的,并且她开始出
离我执的枷锁。她被留在这样一种状态中,她可以有自我如果她这样选择的话——既可
以是她自己的也可以是其它她想要的——她也可以是无我的。其结果是:纯然的觉知。
密宗也不是容易做到的。

内观禅修\myfootnote{Vipassana:毗婆舍那,内观。}是最古老的佛教禅修方法。此方
法直接来源于《{\it 念住经}》\myfootnote{Satipatthana Sutta},该经文由佛陀本人
现身说法。内观是一种对正念或觉知的直接而渐近的培养方法。它经过长年的修行
一点一点的行进。修炼者的注意力被小心地导向对自身存在的某个方面的强烈的审
查中。禅修者通过训练来注意到越来越多人生体验的涌流。内观是柔和的技法,但
它也是非常、非常彻底的方法。它是古老的而且编集成典的训练你内心的系统,是
为了使你越来越觉知你自己人生经验的一系列的练习。它是留心的倾听,具念的
观见,和小心的求证。我们将学会敏锐的嗅闻,全然的触摸,并真正注意发生在这些
经验中的变化。我们将学会倾听我们自己的思想而不深陷其中。

内观禅修的目标是学会看见现象的无常、苦、无我\myfootnote{impermanence、%
unsatisfactoriness、selflessness:无常、苦、无我,三谛。}的真相。我们认为我们
已经是这样做了,但那只是一个错觉。这从\1我们对自己人生经验持续发展的涌流很
少注意就像我们睡着了一样的这个事实就可以看出来。我们甚至对于我们没有注意的
这个事实没有足够注意以意识到它。这又是另一对矛盾。

通过正念的过程,我们慢慢开始对我执形象下的真实的自己变得觉知。我们开始认识
到人生的真相。它不只是一系列的起伏,棒棒糖(奖励)和打手腕(惩罚)。这只是
一个幻象。人生有更深的质感,如果我们愿意去观察,并且以正确的方式去观察。

内观是一种心理训练的形式,它将教你以全新的方式来体验世界。你将第一次学到对
于你、你的周围和你的内心什么正在真正发生。它是一个自我发现的过程,一种参与
式的调查,你在参与其中的同时观察你自己的经验。这个实践必须用以下的态度来开
展:“不必介意你之前被教授的东西。忘记那些理论、偏见和成见。我想要理解人生的
真谛。我想要知道活着的经验的真相是什么。我想要领悟真实的和最深层次的人生的
性质,而不愿意只是接受他人的解释。我想要自己亲眼看到它。”

如果你以这种态度从事禅修,你将获得成功。你将会发现你自己客观地观察事物,如
其所是地观察它们一刻接一刻的流动和变化。此时人生呈现出一种难以置信的无法用
言语描述的丰富多采。它只能被体验。

内观禅修的巴利文为{\it vipassana bhavana}。{\it Bhavana}源于{\it bhu}这个
词根,意为成长或成为。因而{\it bhavana}意味着培养,而且这个词总是用于关于内
心的方面;所以{\it bhavana}意指心理的培养。{\it Vipassana}派生自两个词根。
{\it Passana}意为看见或感知。{\it Vi}是一个具有复杂含意的前缀,大致可以\1译为
“以一种特殊的方式”和{\it 进入}或{\it 通过}“一种特殊的方式”。{\it Vipassana}%
这个词完整的意思是:清晰准确的深入观察事物,视每个部分为独特的组成,并完全
地深入以感知事物最根本的事实。这个过程导致对所审查事物的基本事实的洞见。将
这两个词放在一起,{\it vipassana bhavana}意指:将导致洞见和彻悟的一种特殊方
式的知见作为目标的心理的培养。

在内观中我们培养这个看待人生的特殊方式。我们训练自己如实知见,并且我们称这种
特殊的感知模式为{\it 正念\myfootnote{mindfulness:念、正念、觉知。}}。正念的
过程确实与我们平常所做的大不相同。我们常常忽视在我们面前的真实存在。我们通过
思想和概念的屏幕来看待生活,并误将这些心理的对象当作现实。于是我们深陷无尽的
思想流中以至于现实从身边经过而不自知。我们花费时间沉迷于活动,深陷于对快乐和满足
的无休止的追逐以及对痛苦和不快的永不停息的逃避中。我们花费精力试图使自己感觉
更好,试图埋藏自己的恐惧,并无止尽的寻求安全感。与此同时,真实经验的世界则原封
不动地擦肩而过。在内观禅修中我们训练自己忽视想要更舒适的持续冲动,反而投入到
现实中。讽刺的是真正的平静只有在你停止对舒适的追逐时才会到来——这又是一对矛盾。

当你放松追求舒适的欲望时,真正的成就才会出现。当你放下对满足的匆忙追逐时,人
生真正的美丽才会浮现。当你寻求对没有错觉、包含痛苦和危险的完整现实的认知时,
真正的自由和安全才将会是你的。这不是我们试图向你灌输的教条;它是可观察的现实,
是你可以并且应该亲眼所见的事物。

\1佛教已经有2500年的历史,而任何如此古老的思想体系都会历经岁月形成层层的教条
和仪规。尽管如此,佛教的基本态度仍是彻底基于实证和反权威的。乔达摩佛本人就是
很非正统与众不同的人并且是真正的反传统者。他不是将自己的教导作为一系列教条来
传授,而是作为一系列的建议供个人自己去探究。他邀请每一个人,“自己来看”。他对
他的追随者说的事情之一是,“不要用他人的头脑取代你自己的”。他的意思是,不要轻
易接受他人的言论。自己亲自去验证。

我们希望你将这个态度应用到你在本书中所读到的每一个字上。我们不是在颁布你应该
接受的教条,仅仅因为我们是这个领域的权威。迷信与此毫不相关。这些是经历的现实。
学着按照本书中的教导来调整你感知的模式,你将会自己亲眼见到。这样并且只有这样
才能为你的信仰提供基础。本质上,内观禅修是一种探索的个人发现的修行。

说完这点,我们将在这里给出部分佛教哲学要点的简短概要。我们不会太详尽,因为在
很多其它的书中已经给出了详尽的解释。但因为这些题材是理解内观所必备的,有些内
容不得不提及。

从佛教的观点,我们人类生活在一种非常奇特的方式中。我们将无常的事物视为不变的,
虽然我们身边的事物无时无刻不在变化。变化的过程是不变和持久的。甚至当你在阅读
这些文字时,你的身体正在老去。但是你却没有注意到它。你手中的这本书正在腐朽。
印刷的字体正在褪色,而书页正在变得易碎。你周围的墙壁正在老化。那些墙壁内的分
子正以极快的频率在振动,每个事物都在不断变换,走向分崩离析,\1并且慢慢消溶。
你同样对此没有留意。然后某一天你环顾四周。你的皮肤生起了皱纹关节隐隐作痛。
这本书变成发黄褪色的东西;而这个建筑正土崩瓦解。于是你怀念失去的青春,为你所
有之物失去而哭泣。这些痛苦从何而来?它来自于你的漫不经心。你没能密切得留意生
活。在世界经过时你没能持续地观察它的变化流动。你建立了一系列的心理构造——“我”,
“这本书”,“这个建筑”——而且你认为它们是坚固、真实的实体。你认为它们会永远存在。
而它们却并非如此。但是你现在可以倾听这个持续的变化。你可以学会将你的人生作为
一种不断变化的运动来感知。你可以学会看到所有依缘而生的事物的不停流动。你可以
做到。这只是时间和修行的问题。


\endchapter

\byebye
