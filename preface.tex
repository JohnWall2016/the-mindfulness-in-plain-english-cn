% -*- coding: utf-8 -*-

\input macros

\titlepage
\def\rhead{前言}
\vbox to 6pc{
\bookmark{1}{前言}
\centerline{\titlefont 前言}\vss}
{\topskip 9pc % this makes equal sinkage throughout the Preface
\vskip-\parskip
\twelvepoint
{
\parindent=3pc
\noindent\hang\hangafter-2
\smash{\lower19pt\hbox to 0pt{\hskip-\hangindent\KT{36}在\hfill}}%
{\KT{12}我的经验中},我发现用人们可以理解的方式表达新鲜事物的最有效
的方式就是尽可能使用最简单的语言。我也从教学中学到了:语言越刻板——%
即,事物的描述对人们经验中不可避免的差异缺乏弹性——教学的效果越差。
谁会愿意面对严厉而刻板的说教?特别是在学习新鲜事物时,而且这件事又
是我们在日常生活中一般很少参与的。那样的说教会使禅修,正念的修行,看似
是我们一直无法做到的事情。而本书正是对这个观点对症下药!就其核心而言,
这是一本用通俗的日常语言写成的简单易懂的书——但是在这些书页中,你将发
现丰富的指导用以开启你本人对正念在你生活中的真正力量以及它的许多相关益
处的发现。我是应自己所收到的许多对这么一本入门书的需求而写作本书的。
如果你无法找到师父或富有经验的指导时,你将发现在你自行进行正念的禅修时
本书是特别有用的资源。

}

从智慧出版社首次出版《正念:用通俗的语言诠释》的这二十年中,我们已经看
到正念影响着现代社会和文化的越来越多的方面——教育、心理治疗、艺术、瑜伽、
医疗、以及开始飞速发展的脑科学。而且越来越多的人为着各种原因找到了正念:
减压;改善身体和心理健康;在人际关系、工作、乃至他们的整个人生中变得更
高效、更高明、和更友善。

而我希望,无论是什么机缘使你遇到本书或是本书遇到你,你都将在本书中发
现通向一条无比受益的道路的清楚指引。
\bigskip
\rightline{德宝法师}
}

\vfill\eject\null
\vfill\eject\byebye