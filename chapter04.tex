% -*- coding: utf-8 -*-

\input macros

\beginchapter Chapter 4. 态度

\origpageno=33
\pageno=31

{
\parindent=3pc
\noindent\hang\hangafter-2
\1\smash{\lower19pt\hbox to 0pt{\hskip-\hangindent\KT{36}最\hfill}}%
{\KT{12}近一个世纪},西方科学得到了一个惊人的发现:我们是我们看到的世界
的一部分。正是我们观察的过程改变了我们观察的事物。比如,电子是一个极其
微小的颗粒。在不用仪器的情况下它无法被观测到,而观测的设备决定了观察者
将会看到的。如果你以一种特定的方式观察它,它看上去像一个粒子,一个实心
的小球沿完美的直线路径在四周跳动。当你以另一种方式观察它时,电子呈现波
纹的形态,发光并四处摆动,一点都不像一个坚硬的实体。电子更像是一个事件
而非事物,而观察者正是通过他或她的观察行为参与其中。这种相互作用是无法
避免的。

}

东方科学在很早之前就认识到这个基本原理。内心自身就是一系列的事件,而在
你在每次向内看时就自然而然的参与其中。禅修是一种参与其中的观察:你正在
观察的对象回应这个观察的过程。在这个过程中,你正在观察的对象是你自己,而
你所看到的取决于你所看的方式。因此,禅修的过程是极其微妙的,而其结果完全
取决于禅修者的内心状态。下面将要提及的态度是修行过程中取得成功所必须具备
的;其中大部分之前都有涉及,但我们在这里将其作为实修的法则再次归纳一下:

\medbreak
\1 1){\it 不要怀揣任何预期。}只是放松坐下然后看看发生什么。将整个事件作
为一个实验。带着主动的兴趣投入到测试本身,不要被你对结果的预期分心。而且
不论任何结果不要为其焦虑。让此时的禅修以其自身的速度和自己的方向前进。让
这个禅修来教导你。禅修的觉知寻求对现实的如实知见。无论其是否符合我们的预
期,它需要暂时搁置我们所有的先见和观念。在禅修过程中我们必须将意象、观点
和解读存放在道路的一边。否则我们将被它们绊倒。

\medbreak
2){\it 不要过度用力。}不要强迫任何事情或做出堂皇且夸张的努力。禅修不是攻
城掠地。不存在也不需要凶狠的拼搏。只要让你的努力放松且持续。

\medbreak
3){\it 不要急躁。}因为不赶着去哪里,所以慢慢来。让自己安坐在坐垫上并且就
像你有一整天时间般地打坐。任何真正有价值的事物都需要时间来孕育。耐心、耐
心、耐心。

\medbreak
4){\it 不要执著或拒绝任何事物。}无论是什么,让要来的来,调整你自己以适应
它。如果好的内心意象出现,那很好。如果坏的内心意象出现,那也不赖。平等地
看待它们,并且使自己轻松的应对所发生的一切。不要与你经历的事物抗争,只需
要具念地\myfootnote{mindfully: 以正念、觉知的方式。}观察这一切。

\medbreak
5){\it 顺其自然。}学着与出现的一切变化一起流动。停止焦虑放松下来。


\endchapter

\byebye
