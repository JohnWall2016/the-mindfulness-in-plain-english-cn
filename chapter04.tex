% -*- coding: utf-8 -*-

\input macros

\beginchapter Chapter 4. 态度

\origpageno=33
\pageno=31

{
\parindent=3pc
\noindent\hang\hangafter-2
\1\smash{\lower19pt\hbox to 0pt{\hskip-\hangindent\KT{36}最\hfill}}%
{\KT{12}近一个世纪},西方科学得到了一个惊人的发现:我们是我们看到的世界
的一部分。正是我们观察的过程改变了我们观察的事物。比如,电子是一个极其
微小的颗粒。在不用仪器的情况下它无法被观测到,而观测的设备决定了观察者
将会看到的。如果你以一种特定的方式观察它,它看上去像一个粒子,一个实心
的小球沿完美的直线路径在四周跳动。当你以另一种方式观察它时,电子呈现波
纹的形态,发光并四处摆动,一点都不像一个坚硬的实体。电子更像是一个事件
而非事物,而观察者正是通过他或她的观察行为参与其中。这种相互作用是无法
避免的。

}

东方科学在很早之前就认识到这个基本原理。内心自身就是一系列的事件,而在
你在每次向内看时就自然而然的参与其中。禅修是一种参与其中的观察:你正在
观察的对象回应这个观察的过程。在这个过程中,你正在观察的对象是你自己,而
你所看到的取决于你所看的方式。因此,禅修的过程是极其微妙的,而其结果完全
取决于禅修者的内心状态。下面将要提及的态度是修行过程中取得成功所必须具备
的;其中大部分之前都有涉及,但我们在这里将其作为实修的法则再次归纳一下:

\medbreak
\1 1){\it 不要怀揣任何预期。}只是放松坐下然后看看发生什么。将整个事件作
为一个实验。带着主动的兴趣投入到测试本身,不要被你对结果的预期分心。而且
不论任何结果不要为其焦虑。让此时的禅修以其自身的速度和自己的方向前进。让
这个禅修来教导你。禅修的觉知寻求对现实的如实知见。无论其是否符合我们的预
期,它需要暂时搁置我们所有的先见和观念。在禅修过程中我们必须将意象、观点
和解读存放在道路的一边。否则我们将被它们绊倒。

\medbreak
2){\it 不要过度用力。}不要强迫任何事情或做出堂皇且夸张的努力。禅修不是攻
城掠地。不存在也不需要凶狠的拼搏。只要让你的努力放松且持续。

\medbreak
3){\it 不要急躁。}因为不赶着去哪里,所以慢慢来。让自己安坐在坐垫上并且就
像你有一整天时间般地打坐。任何真正有价值的事物都需要时间来孕育。耐心、耐
心、耐心。

\medbreak
4){\it 不要执著或拒绝任何事物。}无论是什么,让要来的来,调整你自己以适应
它。如果好的内心意象出现,那很好。如果坏的内心意象出现,那也不赖。平等地
看待它们,并且使自己轻松的应对所发生的一切。不要与你经历的事物抗争,只需
要具念地\myfootnote{mindfully: 以正念、觉知的方式。}观察这一切。

\medbreak
5){\it 顺其自然。}学着与出现的一切变化一起流动。停止焦虑放松下来。


\medbreak
6){\it 接受出现的任何事物。}接受你的感受,甚至是那些\1你希望自己不会有的。
接受你的经验,甚至那些你讨厌的。不要因具有人类的瑕疵和缺点而责备自己。学
会将心中所有现象视为完全正常和可以理解的。试着在任何时候对你经验的事物行
使无著的接受。

\medbreak
7){\it 温和地对待你自己。}和善地对待你自己。你可能并不完美的,但你完全无
法回避自己。成为未来的你的过程开始于对现在的你的完全接受。

\medbreak
8){\it 亲自实证。}对任何事物发问。不要认为任何事物是理所当然的。不要轻
信任何事物,仅因它听上去明智而虔诚并且某个圣人如是说。自己亲自去看。这并
不意味着你应该愤世、粗鲁或傲慢。它意味着你应该是实证的。将所有论断纳入你
自己经验的实物测试中,并让其结果成为你通向真理的向导。内观禅修发展自这样
一种内在的渴望,即渴望对真实世界的觉醒和渴望获得对存在的真实结构的可令人
解脱的洞见。这种修行完全基于对觉悟真如的渴望。没有这个渴望,这种修行就是
肤浅表面的。

\medbreak
9){\it 将所有的困难都视为考验。}将摆在眼前的负面情况视为学习和成长的
机遇。不要逃避它,不要责备自己,也不要将你的负担埋入沉默之中。你有一个问
题?很好。更多的修行机会。欢喜的,投入其中,并亲自去探究。

\medbreak
10){\it 不要思索。}你不需要想通每一件事情。逻辑思维不会让你从陷阱中解脱。
在禅修中,通过觉知、通过无语的纯粹留意,内心自然地得到净化。习惯性的思索
对于消除\1那些束缚你的事物不是必需的。所必需的是对这些事物是什么以及它们
如何运作的清楚的、非概念的感知。这个感知足以化解它们。概念和推理只会设立
障碍。不要思考。直接去看。

\medbreak
11){\it 不要老想着与他人的差异。}人与人之间确实存在不同,老是念叨着它们
是一个危险的过程。如果不小心处理,这将导致自我主义\myfootnote{egotism}。
一般人的思考中充满贪欲、妒忌和傲慢。一个男人看到街上的另外一个男人会立即
想到,“他比我英俊”。这样想的直接结果是羡慕或羞愧。一个女孩看到另一个
女孩会想,“我比她美丽”。这样想的直接结果是高傲。这种攀比是一种心理习惯,
而它直接导致或这或那的恶感:贪欲、羡慕、傲慢、妒忌或仇恨。它是一种不善巧
的内心状态,但我们往往就是这样做的。我们与他人比较长相、我们的成功、成就、
财富、资产或智商,而这所有的一切导致同样的状态——疏远、人与人的隔阂、以及
敌意。

禅修者要做的是通过全面地审视这种不善巧的习惯来结束它,并用别一种善巧的
习惯来取代它。禅修者注意的不是人与人之间的差异,而是训练他或她自己留意
人与人之间的共性。她将注意力集中于众生共同的因素上、那些拉近彼此间距离
的事物上。如果她有比较也只会带来亲近感而非疏远感。

呼吸是一个普遍的过程。所有的脊椎动物都以本质上相同的方式呼吸。所有的生物
都以这样或那样的方式与环境交换气息。这是呼吸被选作禅修的一个焦点的原因之
一。禅修者被建议探索他或她自己的呼吸以作为证悟我们与其它生命体间的固有的
关联性。这\1并不意味我们忽视周围的不同。不同确实存在。这只意味着我们不强
调差异而强调那些我们共同的普遍的因素。

所建议的过程如下:当我们作为禅修者感知任意感观对象时,我们不以往常的自
我主义的方式停留于该对象本身。我们更应该审视感知本身的这个过程。我们应该观
察这个对象对我们的感觉和感知的作用。我们应该观察出现的感受和继之而起的心
理活动。我们应留意作为结果发生在我们自己意识中各种变化。通过观察所有的这些
现象,我们将会觉察到我们所见事物中的普遍性。初始的觉知会引发乐、苦、舍受%
\myfootnote{乐、苦、舍受: 快乐的、不快乐的、不苦不乐的感受。}。这是一个普
遍的现象,在其他人心中发生的和我们自己经历的一样,并且我们应该清楚地看到
这一点。这些感受之后不同的反应将会产生。我们可能感受到贪念、欲望、或妒忌。
我们可能感受到害怕、焦虑、不安、或厌烦。这些反应也是普遍的。我们应该轻松地
留意它们并且由己及人地推而广之。我们应该认识到这些反应是正常的人类反应,
并且可能发生在任何人身上。

这种比较方式的实修一开始可能会有勉强和造作的感觉,但它一点也不比我们平常
所做的更不自然。它只是有点陌生。通过实修,这种习惯的模式将取代我们平常自
我主义的比较习惯并最终感觉更为自然。结果我们将成为非常善于理解人情世故的
人。我们不再因他人的“过失”而烦恼。我们与所有生命向着和谐的方向前进。

\endchapter

\byebye
