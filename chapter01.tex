% -*- coding: utf-8 -*-

\input macros

\beginchapter Chapter 1. 禅修:为何修禅?

\origpageno=1
\pageno=1

{
\parindent=3pc
\noindent\hang\hangafter-2
\1\smash{\lower19pt\hbox to 0pt{\hskip-\hangindent\KT{36}禅\hfill}}%
{\KT{12}修\myfootnote{meditation:禅修,冥想。}并非易事}。它需要时间和精力。
它还需要勇气、决心、自律。它需要许多我们一般视为不舒服的并想尽可能避免的个
人品质。我们将所有这些品质总结为{\it 进取心\myfootnote{gumption:勇气
和决心,进取而不执著。}}。禅修需要进取心。放松下来看看电视比它容易得多。
那为何还要修禅?为什么将你本可以外出尽情享受的这些时间和精力白白浪费?
为什么?原因很简单。因为你是人类。就是因为你是人类这个简单的事实,你
发现自己是那个无法轻易挥之而去的与生俱来的无法满足于人生的继承者。你可
以压抑它一时不被自己察觉;你可以持续数小时转移自己的注意力,但它总会
转土重来,并且常常出乎你的意料。突然之间,看似毫无原由的,你坐起来,审
视现状,然后认识到你生活中的实际状况。

}

如此这般,然后你突然意识到你现在的整个人生只是在混日子。你保持着光鲜
的外表。你能设法维持生计并且从外表看上去还不错。但那些绝忘的时期,那
些你感到所有事物都向你塌陷的时间——你对这些却守口如瓶。其实你是一团糟,
对此你心知肚明。但是你掩饰得很好。同时,在这所有的外表之下,你只知道
一定存在某种其它的生活方式,一种更好的看待世界的方式,一种更全然的感
触人生的方式。你\1偶尔会时不时地切入其中:你获得一份好的工作时;你坠
入爱河时;在你赢得比赛时。一时间,事物变得与众不同了。生活披上了一层
浓烈而清楚的外衣,它使得所有坏的时光和乏味烟销云散。你经验的整个质地
改变了,然后你对自己说,“好的,我现在成功了;我现在将会变得快乐。”但之
后这种情况也消逝了,就像空中的烟雾。留给你的仅是记忆——以及模糊地觉察
到哪里出问题了。

你感觉到生活中真实存在完全另外一个深度和敏感度的领域;但因某种原因你就
是没有看到它。你发现自己感觉被遮挡。你觉得因为某种感觉的棉花使你与体
验的甜美相隔绝。你没有真正在感触生活。你没有再一次“成功”。之后甚至那
个模糊的觉知也日渐消退,于是你又回到同一个旧的现实中。世界看起来仍是
原来那个脏乱的场所。它是情感的过山车,而你大多数时间花在了坡道的底部,
向往着高处。

那么是你出了什么问题?难道你异于常人?不。你只是人类。而你深受折磨的是
每个人类都会感染的相同疾病。它是我们所有人内在的怪兽,而它有很多武器:
周期性紧张,缺乏对他人真正的同情,包括与你最亲近的人,受阻的感受以及
情绪的麻木——很多,很多种武器。我们没有人能完全摆脱它。我们会否认它。我
们试图压抑它。为躲避它我们建立了整套文化,假装它不在那里,用目标、计划
以及对身份的关切来转移我们自己的注意。但它从未离去。它是每个念头和感知
中持继的暗流,是在我们脑后不停絮叨的细小声音,“还不足够好。还需要更多。
必须使其更好。必须更好。”它是怪兽,以不易察觉的方式在每处显现的怪兽。

去参加聚会。听听那些欢笑声,那些表面是欢愉底下是恐惧的尖利声音。感受那
紧\1张、压力。没有一个人真正放松。他们只是在装腔作势。去看球赛。注意看台
上的球迷。注意那阵阵无理性的愤怒。注意人们在热情和团队精神伪装之下洋溢
的无法控制的沮丧。以团队忠诚之名的嘘声、贺倒彩和失控的自我主义。看台上
的醉酒、打斗——这是人们不顾一切地试图释放内心的压力;这些人内心并不平静。
留心看电视上的新闻。留心听流行音乐的歌词。你会发现相同主题在变化中不断
重复:妒忌、痛苦、不满和压力。

生活似乎是一场不断的抗争,是对抗难以置信的命运的巨大努力。而对于这些不
满我们的解决方案是什么?我们陷入“只要”综合症。只要我有更多的钱,我就会
开心。只要我能找到真正爱我的人;只要我能减掉二十磅;只要我有彩色电视机、
热水浴池和卷发;如此这般无穷无尽。这些无用之物从何而来,更重要的是,我
们能怎样应对它呢?它来自于我们自己内心的疾病。它是一系列深层次的、细微的
并且普遍的心理习惯,是我们一点点缠绕而成的戈耳迪之结\myfootnote{a Gordian
knot:戈耳迪之结。},我们也只能以相同的方式一次一段地解开它。我们可以
调节自己的觉知,挖掘每一个独立的片段,将其带入光明之中。我们可以慢慢地,
一次一点地使无意识变得有意识。

我们经验的本质是变化。变化是持续不断的。生命一刻接一刻地悄悄流逝,而且
它每一刻都与众不同。不停的变动是感知世界的本质。一个念头才跃入你的脑中,
半秒之后它就离开了。接着另一个念头出现,继而同样又离开。一个声音进入你
的耳朵,接着是寂静。睁开你的眼睛世界映入你的眼帘,转瞬之间它就不见了。人
们进入你的生活然后离开。友人离去,亲人逝去。你的财富上升,它们时而又下
降。你有时取得胜利,\1但又时常失败。它永无休止:变化,变化,变化;没有哪两
个时刻曾是相同的。

不断变化并没有什么不对。这是宇宙的本性。但是人类的文化教给我们一些对这
个不断流动的奇怪反应。我们将经验分类。我们试图将每个感知,在这个不断流
动中的每个心理变化,置于三个心理设定的鸽巢之一中:它是善、恶或无记的%
\myfootnote{good,bad,or neutral;好、坏、不好不坏。}。然后基于这分类,我们
感知到一系列固定的习惯性的心理反应。如果一个特定的感知被标为“善的”,我
们将试图使时间凝固于此。我们抓住那个特定的念头,反复琢磨它,使劲把持它,
并且我们试图不让它逃走。当这样做不凑效时,我们会竭尽所能地努力重复导致
这个念头出现的经验。让我们将这种心理习惯称为“执取”\myfootnoteaq{grasping:%
执取,攥取。}。

在内心的另一侧放着被标记为“恶的”盒子。当我们感知到某事物是“恶的”时,我
们试图将它推向一边。我们试图否认它,排斥它,并且尽力消除它。我们与自己
的经验相抗争。我们逃离自身的一部分。让我们将这种心理习惯称为“瞋拒”
\myfootnoteaq{rejecting:瞋拒,排斥。}。在这两种反应之间放着“中性的”盒子。
我们将既不算好又不算坏的经验放在这里。它们是无激情的、中性的、无趣味的。
我们将这样的经验塞入中性的盒子以使我们能忽略它继而将我们的注意力投入到
真正令人激动的地方,即我们贪与瞋\myfootnote{desire and aversion:贪与瞋。}%
的无尽轮回中。所以“中性”类的经验被剥夺了公平地分享我们注意力的份额。让
我们将这种心理习惯称为“无明”\myfootnoteaq{ignoring:无明,忽略。}。所有
这些疯狂行为的直接结果是持续的一成不变的徒劳奔忙,无止尽地追逐快乐,无
止尽地逃离痛苦,而且无止尽地忽略我们百分之九十的经验。然后我们奇怪人生
为何感觉如此无聊。总之一句话,这个系统行不通。

无论你多么努力地追逐快乐和成功,你总会有失败的时候。无论你逃避得多快,总
会有痛苦追上你的时候。而在这两者之间,生活\1无聊得让你尖叫。我们的内心
充满了意见和指责。我们围绕自己筑起高墙使自己深陷自身喜恶的牢狱中。我们
深受其苦。

“苦”\myfootnoteaq{suffering:苦,痛苦。}在佛教的思想中是一个重要的词汇。
它是一个关键的术语并且应被透彻地理解。苦对应的巴利语是“{\it dukkha}\/”,而
它不仅仅是指肉体的极度痛苦。它是指深层次的、不易察觉的不满感,它是每个
心识刹那\myfootnote{mind moment:心识刹那,意识的瞬间。}的一部分而它是心
理单调乏味的运作方式的直接结果。人生本苦,佛陀如是说。乍看上去,这种说法
看似极度的病态和悲观。它甚至看似不真实。毕竟,我们开心的时候还是很多的。
难道不是吗?不,并非如此。它只是看似开心。拿任何你真正觉得满足的时刻来仔细
观察。在这个喜悦之下,你将发现微妙的、遍及的潜在紧张情绪:无论这一刻多么
伟大,它终将结束。无论你刚刚得到多少,你将不可避免地失去它们或将终其一身
守护所得和谋划得到更多。而最终,你将面临死亡,你终将失去一切。这一切只是
昙花一现。

听上去非常无望,不是吗?庆幸的是,它并非如此——完全不是这样。只有当你以普
通的心理视角——这个视角正是单调乏味的心理机制运作的角度——看待它时,它才看
似无望。在这之下存在另一个视角,一个完全不同的看待世界的方式。它是一个
运作的层面,在此之中内心不会试图凝固时间,不会执著于流动中的经验,也不会
阻断和忽略它们。它是一个超越好坏,超越乐苦的经验层面。它是一个感知世界的
美妙方式,而且它是一个可以习得的技巧。它虽然不容易,但是可以学会。

快乐和平静\myfootnote{快乐是在善的感受之中,平静是对恶的感受的消除。}
是人类存在中真正重要的问题。那是我们所有人所追寻的。它常常有点难以\1看
清楚,因为我们将这些基本目的掩盖在层层的表面目标之下。我们需要食物、财
富、性、娱乐和名望。我们甚至对自己说“快乐”这个概念太抽象:“你看,我是一
个实际的人。只要给我足够的钱我就能够买到我需要的所有快乐。”但很不幸,这
个态度是行不通的。逐一细查这些目标,你会发现它们都是表面的。你需要食物。
为什么?因为我饿。你饿——那又怎样?好吧,如果我吃东西,我就不会饿了,并且
之后会觉得舒服。啊哈!“觉得舒”:这才是正题。我们真正追寻的不是那些表面目
标;它们只是达到目的的手段。我们真正追求的是随着欲望满足而带来的解脱感
\myfootnote{我们追求的一方面是消除不舒服感;另一方面是保持快感。但这两
者每一次的实现都是短暂不可持续的。}。解脱、放松以及紧张的终结。平静、快
乐——仅此而已。

那么什么是快乐?对我们大多数人而言,完美快乐的观念是拥有我们想要的一切
并且控制所有的一切,扮演凯撒的角色,让整个世界按我们的每个念头轻快地舞动。
这个观念再一次行不通。看看历史中事实上拥有这种权力的人们。他们并不是快乐
的人。当然,他们也无法心平静气地对待自己。为什么是这样?因为他们立志完全
彻底地控制这个世界,但他们却无法做到。他们想要控制所有人,但是总有人拒绝
被控制。这些有权势的人无法控制星辰。他们仍然会生病。他们仍然会死亡。

你甚至无法得到你想要的一切。它是不可能的。庆幸的是,还有另外一种选择。你
可以学会控制你的心,以走出贪与瞋的无尽轮回。你可以学会不去执著你想要的,
学会认识欲望而不被欲望控制。这并不意味着要你躺在大街上邀请每一个人从你身
上走过。它意味着你继续着看上去正常的人生,但是以全新的视角生活。你做\1一
个人该做的,但是你将摆脱自身欲望的沉迷而强迫的驱使。你需要某事物,但你不
必要对它紧追不舍。你害怕某事物,但你不必要对它瑟瑟发抖。这种心理的修练是
非常困难的。它需要花相当长的时间。但是既然控制一切外物不可能实现;困难总
比不可能更可取。

可是等一下。平静和快乐!难道人类文明不正是为此吗?我们建设摩天大楼和高速
公路。我们有带薪休假、电视娱乐;我们提供免费医疗和病假,社会保险和社会福
利。所有的这些都着眼于提供某种平静和快乐的举措。但是心理疾病的比率仍在
稳步攀升;而犯罪率也在更快上升。街上充斥着好斗和不稳定的个体。将你的手臂
伸出你安全的房门,有人就很有可能偷走你的手表!一定是那里出问题了。快乐的
人是不会偷盗的。一个对自己心平静气的人是不会觉得有杀念的。我们愿意认为我
们的社会在运用人类知识的各个方面来获取平静和快乐,但事实并非如此。

我们才刚刚认识到:我们以更深的情感和心灵方面作为代价过度地发展了存在的物
质方面,并且我们正为这个错误付出代价。讨论今天美国社会道德和精神力量的堕
落是一回事,而真正为此做点什么则是另一回事。改变开始的地方就在我们的内心。
仔细地、诚实地和客观地向内看,我们每个人将看到“罪由我起”和“我是疯癫”的时
刻。我们将学会看清楚这些时刻,不带自责地非常清楚地看待它们,然后我们将踏
上走出这种状态的道路。


\endchapter

\vfill\eject\byebye
